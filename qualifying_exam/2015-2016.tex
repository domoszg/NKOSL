\documentclass[a4paper,11pt]{exam}
\usepackage[left=1.5cm, right=1.5cm, top=3cm]{geometry}
\usepackage[utf8]{inputenc}
\usepackage[T1]{fontenc}
\setlength\parindent{0pt}
\renewcommand{\familydefault}{\sfdefault}
\newcommand{\shell}[1]{\texttt{#1}}

\printanswers

\usepackage{xpatch}
\xpatchcmd{\oneparchoices}{\penalty -50\hskip 1em plus 1em\relax}{\hfill}{}{}
\xpatchcmd{\oneparchoices}{\penalty -50\hskip 1em plus 1em\relax}{\hfill}{}{}

\begin{document}
\firstpageheader
	{}
	{{\large \textbf{Napredno korištenje OS Linux}}\\
		\textbf{Prijemni ispit - 9. ožujka 2016.}}
	{}
\footer
	{}{}{\thepage}
	
Ime i prezime: \fillin[][7cm] \hfill JMBAG: \fillin[][5cm]\\

\begin{questions}
	\question
	Jezgra operacijskog sustava je 
	
	\begin{oneparchoices}
		\choice ljuska (\textit{shell})
		\CorrectChoice kernel 
		\choice bootloader  
		\choice terminal 
	\end{oneparchoices}
	
	\question
	Što ne vrijedi za operacijski sustav Linux?
	
	\begin{oneparchoices}
		\choice Višekorisnički je
		\choice Pripada Unix obitelji
		\choice Otvorenog je koda
		\CorrectChoice Baziran je na NT kernelu
	\end{oneparchoices}
	
	\question
	Koji sustav za upravljanje paketa se nalazi na distribucijama baziranima na Debianu?
	
	\begin{oneparchoices}
		\choice pacman 
		\choice RPM 
		\CorrectChoice dpkg 
		\choice upkg 
	\end{oneparchoices}
	
	\question
    Naredba \shell{cd \textasciitilde/..} mijenja radni direktorij u
    
	\begin{oneparchoices}
		\choice direktorij naziva \shell{..} \\
        \CorrectChoice direktorij \textit{iznad} matičnog direktorija u hijerarhijskoj strukturi \\
        \choice korijenski (\textit{root}) direktorij \\
        \choice matični direktorij 
	\end{oneparchoices}

	\question
    Pretpostavite da ste \shell{root}. Nalazite se u korijenskom direktoriju \shell{/} i izvršavate naredbu \shell{cd ..} 
	
	\begin{oneparchoices}
        \choice Naredba se ne može izvršiti \\
        \CorrectChoice Naredba će se izvršiti i radni direktorij ostaje nepromijenjen \\
        \choice Naredba će se izvršiti i matični direktorij postaje radni direktorij \\
        \choice \shell{root} može sve pa će se naredba izvršiti i korisnik će se pomaknuti u nepostojeći direktorij
	\end{oneparchoices}
	
	\question
	Datoteka čiji naziv počinje s \shell{.} predstavlja
	
	\begin{oneparchoices}
		\choice obrisanu datoteku
		\choice novu datoteku
		\CorrectChoice skrivenu datoteku
		\choice privremenu datoteku
	\end{oneparchoices}
	
	\question	
	Naredba \shell{file} određuje tip datoteke iz
	
	\begin{oneparchoices}
		\choice ekstenzije 
		\choice veličine datoteke 
		\CorrectChoice sadržaja datoteke 
		\choice podatka o zadnjem korištenju 
	\end{oneparchoices}
	
	\question
	\shell{\$ ls -l file1} \\
	\shell{-rw-r--r-- 1 user users 0 Jan 4 23:19 datoteka.txt} \\
	Koje je značenje prvog polja u gornjem ispisu? (\textit{moda} datoteke)
	
	\begin{oneparchoices}
		\choice Vlasnici datoteke
		\choice Lokacija datoteke
		\choice Prva 3 okteta sadržaja
		\CorrectChoice Dozvole pristupa
	\end{oneparchoices}
	
	\question
	\shell{\$ ls -l file1} \\
	\shell{-rw-r--r-- 1 user users 0 Jan 4 23:19 datoteka.txt} \\
	Što predstavlja broj iza moda datoteke u gornjem ispisu?
	
	\begin{oneparchoices}
		\choice Ukupan broj linkova 
		\CorrectChoice Broj hard linkova 
		\choice Broj pristupa datoteci 
		\choice Broj kopija datoteke 
	\end{oneparchoices}
	
	\question
	Koja od sljedećih naredbi kopira direktorij \shell{izvorno} i sav njegov sadržaj na lokaciju \shell{novo}?
	
	\begin{oneparchoices}
		\CorrectChoice \shell{cp -R izvorno novo} \\
		\choice \shell{cp izvorno novo} \\
		\choice \shell{mv izvorno novo} \\
		\choice \shell{cp -S izvorno novo}
	\end{oneparchoices}
	
	\question
	Naredba \shell{ln dir1 file1}
	
	\begin{oneparchoices}
		\choice kopira \shell{dir1} na lokaciju \shell{file1} \\
		\CorrectChoice stvara hard link \shell{file1} na direktorij \shell{dir1} \\
		\choice mijenja dozvole za \shell{dir1} i \shell{file1} \\
		\choice stvara symbolic link \shell{file1} na direktorij \shell{dir1}
	\end{oneparchoices}
	
	\question
	Koji od sljedećih modova će omogućiti vlasniku - korisniku datoteke da je \textit{izvrši}?
	
	\begin{oneparchoices}
		\choice 4400
		\choice 440
		\CorrectChoice 540
		\choice 640 
	\end{oneparchoices}
	
	\question
	Što općenito ne vrijedi za TAR arhivski format?
	
	\begin{oneparchoices}
		\CorrectChoice Komprimira sadržaj datoteka \\
		\choice Čuva dozvole pristupa \\
		\choice Čuva strukturu direktorija \\
		\choice Čuva MAC vremena
	\end{oneparchoices}

	\question
	Direktorij \shell{/etc/skel} sadrži
	
	\begin{oneparchoices}
		\choice informacije o korisnicima 
		\choice konfiguraciju mreže 
		\choice predložak korijenskog direktorija \\
		\CorrectChoice predložak matičnih direktorija 
	\end{oneparchoices}
	
	\question
	Datoteka \shell{/etc/passwd} na modernim sustavima sadrži
	
	\begin{oneparchoices}
		\choice korisničke lozinke 
		\CorrectChoice popis korisnika na sustavu
		\choice popis računala kojima korisnik smije pristupiti
		\choice datoteka je obično prazna 
	\end{oneparchoices}
	
	\question
	Koji podatak se sigurno neće naći u datoteci \shell{/etc/passwd}?
	
	\begin{oneparchoices}
		\choice Broj telefona korisnika
		\CorrectChoice Popis grupa kojima korisnik pripada
		\choice Putanja do login ljuske \\
		\choice UID korisnika
	\end{oneparchoices}
	
	\question
	Želite instalirati MATLAB 2015a na računalo i instalacijski program vam nudi odabir odredišnog direktorija za programske datoteke. Instalacija MATLAB-a zauzima oko 9 GB. Koju zadanu lokaciju je program predložio?
	
	\begin{oneparchoices}
		\choice \shell{/home/ivica/MATLAB} 
		\CorrectChoice \shell{/opt/MATLAB} 
		\choice \shell{/usr/lib/MATLAB} 
		\choice \shell{/usr/share/MATLAB} 
	\end{oneparchoices}
	
	\question
	Izmjenjivi diskovi (npr. USB memorije) se najčešće montiraju (\textit{mountaju}) na lokaciji
	
	\begin{oneparchoices}
		\CorrectChoice \shell{/media}
		\choice \shell{/usr}
		\choice \shell{/home}
		\choice \shell{/etc} 
	\end{oneparchoices}

	\question
	Naredba \shell{ls -l /dev/sd?} popisuje datoteke koje na modernim distribucijama predstavljaju
	
	\begin{oneparchoices}
		\CorrectChoice SCSI i SATA diskove 
		\choice floppy uređaje 
		\choice zvučne kartice 
		\choice IDE diskove
	\end{oneparchoices}

	\question
	Blok datoteka \shell{/dev/sda1} predstavlja montiranu particiju. Koja će naredba dati informaciju o zauzeću datotečnog sustava na toj particiji?
	
	\begin{oneparchoices}
		\choice \shell{du -s /dev/sda1}
		\choice \shell{ls -ls /dev/sda1}
		\CorrectChoice \shell{df}
		\choice \shell{stat /dev/sda1}
	\end{oneparchoices}
	
	\question
	U matičnom direktoriju se nalazi direktorij \shell{data}. Koja naredba vraća ukupno diskovno zauzeće datoteka unutar tog direktorija i svih njegovih poddirektorija?
	
	\begin{oneparchoices}
		\CorrectChoice \shell{du -s \textasciitilde/data}
		\choice \shell{ls -ls \textasciitilde/data}
		\choice \shell{df}
		\choice \shell{stat \textasciitilde/data}
	\end{oneparchoices}
	
	\question
	Putanje u kojima ljuska traži izvršne datoteke programa koje korisnik poziva nalaze se u varijabli okruženja
	
	\begin{oneparchoices}
		\choice \shell{\$PROGRAMS}
		\choice \shell{\$LIST}
		\CorrectChoice \shell{\$PATH}
		\choice \shell{\shell{\$SHELL}}
	\end{oneparchoices}
	
	\question
	Dokumentacija programa \shell{ln} dobiva se pozivanjem
	
	\begin{oneparchoices}
		\choice \shell{support ln}
		\choice \shell{ln -?}
		\choice \verb|ln --all|
		\CorrectChoice \shell{man ln}
	\end{oneparchoices}
	
	\question
	Izbaci uljeza  
	
	\begin{oneparchoices}
		\choice nano 
		\choice vim  
		\choice emacs 
		\CorrectChoice gcc 
	\end{oneparchoices}
	
	\question
	Kojom naredbom biste projerili sve aktivne procese na sustavu 
	
	\begin{oneparchoices}
		\choice info 
		\CorrectChoice ps 
		\choice jobs 
		\choice stat 
	\end{oneparchoices}
	
	\question
	Kombinacija tipki CTRL + c na uobičajenim konfiguracijama šalje signal
	
	\begin{oneparchoices}
		\choice SIGHUP (1) 
		\CorrectChoice SIGINT (2) 
		\choice SIGKILL (9) 
		\choice SIGUSR1 (10) 
	\end{oneparchoices}
	
	\question
	Koji signal bez odgode prekida rad procesa?
	
	\begin{oneparchoices}
		\choice SIGHUP (1) 
		\choice SIGINT (2) 
		\CorrectChoice SIGKILL (9) 
		\choice SIGUSR1 (10) 
	\end{oneparchoices}
	
	\question
	Cjevovod \shell{2>} preusmjerava
	
	\begin{oneparchoices}
		\choice \shell{stdout} na \shell{stderr}
		\choice \shell{stderr} na \shell{stdout}
		\choice \shell{stdout} u datoteku
		\CorrectChoice \shell{stderr} u datoteku
	\end{oneparchoices}
	
    
\vspace{1em}
\uplevel{
	Za sljedeća pitanja odredite naredbu koja se prosljeđuje naredbi \shell{sed -r} kako bi se postigla zadana promjena nad stringovima.
	
	\textbf{Primjer:} \, \shell{acd12a} $\longrightarrow$ \shell{acdaaa}\\
	Odgovor: \shell{'s/[[:digit:]]/a/'}
}

	\question Jedan samoglasnik s kraja retka se briše; \shell{Minervae} $\longrightarrow$ \shell{Minerva}
	
	\begin{choices}
		\choice \verb|'s/^(.*)[aeiouAEIOU]/\1/'|
		\choice \verb|'s/^(.*)[aeiouAEIOU]./\1/'|
		\CorrectChoice \verb|'s/^(.*)[aeiouAEIOU]$/\1/'|
		\choice \verb|'s/^(.*)[aeiouAEIOU]*/\1/'| 
	\end{choices}
	
	\question Znakovi na 3. i 4. poziciji postaju veliki; \shell{iowa} $\longrightarrow$ \shell{ioWA}

	\begin{choices}
		\CorrectChoice \verb|'s/^(..)(.{2})/\1\U\2/'|
		\choice \verb|'s/(..)(..)/\1\U\2/'|
		\choice \verb|'s/^..(..)/\1\U\2/'|
		\choice \verb|'s/$(..)(..)/\1\U\2'|
	\end{choices}
\end{questions}
\end{document}
