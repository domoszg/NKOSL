\documentclass[12pt,a4paper]{article}

\usepackage[croatian]{babel}
\usepackage[utf8]{inputenc}

\usepackage[margin=2cm]{geometry}
\usepackage[colorlinks=true,urlcolor=black]{hyperref}
\pagenumbering{gobble}

\begin{document}
	\title{Domaća zadaća 4}
	\date{\vspace{-5ex} 24.05.2015.}
	\maketitle
	

\section*{Zadatak}

Instalirati i konfigurirati SSH i FTP server.\\

Pristup SSH serveru je dozvoljen iz lokalne mreže 10.0.0.0/8 i s localhosta 127.0.0.1/32 na portu 1001. Lokalni korisnici se autentificiraju RSA ključevima, a autentifikaciju lozinkom je potrebno onemogućiti. SSH server mora imati omogućen X11 forwarding.\\
Stvoriti nove korisnike \texttt{dino}, \texttt{dominik}, \texttt{nino} i \texttt{studenti}. Korisnici moraju imati home folder. Za svakog korisnika generirati RSA ključeve i konfigurirati ih za pristup SSH serveru.\\

Podesiti FTP server za anonimni pristup za čitanje direktorija \texttt{/var/ftp/public}. Lokalni korisnici \texttt{dino}, \texttt{dominik}, \texttt{nino} i \texttt{studenti} se mogu autentificirati lozinkom i pristupiti cijelom filesystemu. FTP server ih nakon autentifikacije postavlja u home folder. Ostalim korisnicima na računalu treba onemogućiti autentificirani pristup. Pristup serveru je dozvoljen s bilo koje adrese kroz port 21 u pasivnom načinu rada.\\

Kao rješenje šaljete tar arhivu sa svim korištenim konfiguracijskim datotekama i eventualnim dodatnim skriptama koje ste napisali.

\end{document}