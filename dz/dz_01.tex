\documentclass[12pt,a4paper]{article}

\usepackage[croatian]{babel}
\usepackage[utf8]{inputenc}

\usepackage[margin=2cm]{geometry}
\usepackage[colorlinks=true,urlcolor=black]{hyperref}
\pagenumbering{gobble}

\begin{document}
	\title{Domaća zadaća 1}
	\date{\vspace{-5ex} 08.03.2015.}
	\maketitle
	Prije rješavanja DZ potrebno je osigurati sljedeće uvjete na računalu ili virtualnom stroju na kojem radite:
	\begin{itemize}
		\item Instaliran najmanje jedan tvrdi disk s jednom particijom tipa \texttt{ext2/ext3/ext4} te jednom tipa \texttt{swap}.\\ {\small (Na disku može biti i više particija koje nećete koristiti u ovoj DZ.)}
		\item Na ext particiju instalirana neka distribucija Linuxa s jednim od bootloadera koji se spominjao na predavanju.
	\end{itemize}
	\section*{Zadatak}
	Potrebno je osigurati bootanje i mountanje ext i swap particije koristeći jednu od metoda za \emph{persistent block device naming} opisanih na wiki stranicama \url{https://wiki.archlinux.org/index.php/Persistent_block_device_naming}. Kao rješenje domaće zadaće šaljete sljedeće datoteke:
	\begin{itemize}
		\item \texttt{/etc/fstab}\\
			u kojoj ste prema jednoj od smjernica na gornjem linku osigurali mountanje ext i swap particije na ispravne lokacije i s ispravnim parametrima
		\item Konfiguracijsku datoteku bootloadera\\
			u kojoj se također vidi da koristite jednu od \emph{persistent naming} smjernica
		\item[] Datoteka koju šaljete ovisi o bootloaderu kojeg koristite:
		\item[] \begin{tabular}{p{4cm} p{5cm}}
					\textbf{Bootloader} & \textbf{Datoteka za slanje} \\
					GRUB 2 & \texttt{/boot/grub/grub.cfg}\\
					GRUB Legacy & \texttt{/boot/grub/menu.lst}\\
					Syslinux & \texttt{/boot/syslinux/syslinux.cfg}
				\end{tabular}
		\item Ispis naredbe \texttt{lsblk -f} spremljen na lokaciju \texttt{/var/block.log}
	\end{itemize}
	
	Datoteke pospremite u tar arhivu pri tome čuvajući apsolutne staze datoteka. (Dakle, struktura tar arhive mora biti: \texttt{folder boot $\rightarrow$ folder grub/syslinux $\rightarrow$ datoteka} i tako za sve tražene datoteke.)\\
	
\end{document}