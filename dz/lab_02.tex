\documentclass[12pt,a4paper]{article}

\usepackage[croatian]{babel}
\usepackage[utf8]{inputenc}

\usepackage[margin=2cm]{geometry}
\usepackage[colorlinks=true,urlcolor=black]{hyperref}
\usepackage{datetime}
\pagenumbering{gobble}

\renewcommand{\dateseparator}{.}
\newcommand{\todayiso}{\twodigit\day \dateseparator \twodigit\month \dateseparator \the \year}

\begin{document}
    \title{Laboratorijska vježba 2\\{\large Napredno korištenje operacijskog sustava Linux}}
    \date{\todayiso}
	\date{\vspace{-5ex} }
	\maketitle
	

\section{Zadatak}

Instalirati i konfigurirati SSH i FTP server.\\

Pristup SSH serveru je dozvoljen iz lokalne mreže 10.0.0.0/8 i s localhosta 127.0.0.1/32 na portu 1001. Lokalni korisnici se autentificiraju RSA ključevima, a autentifikaciju lozinkom je potrebno onemogućiti. SSH server mora imati omogućen X11 forwarding.\\
Stvoriti nove korisnike \texttt{dino}, \texttt{dominik}, \texttt{nino} i \texttt{studenti}. Korisnici moraju imati home folder. Za svakog korisnika generirati RSA ključeve i konfigurirati ih za pristup SSH serveru.\\

Podesiti FTP server za anonimni pristup za čitanje direktorija \texttt{/var/ftp/public}. Lokalni korisnici \texttt{dino}, \texttt{dominik}, \texttt{nino} i \texttt{studenti} se mogu autentificirati lozinkom i pristupiti cijelom filesystemu. FTP server ih nakon autentifikacije postavlja u home folder. Ostalim korisnicima na računalu treba onemogućiti autentificirani pristup. Pristup serveru je dozvoljen s bilo koje adrese kroz port 21 u pasivnom načinu rada.\\



\section{Zadatak}

Instalirati i konfigurirati NGINX, WSGI and Python stack. \\

Uz zadatak je priložena python poslužiteljska aplikacija HelloWorldServer.py. \\
Direktno korištenje aplikacije je \textbf{python HelloWorldServer.py} te iz drugog terminala pozvati \textbf{curl localhost:8888}.
Napomena: Potrebno je instalirati python-flask paket. \\

Potrebno je konfigurirati Nginx da se aplikaciji može pristupiti na lokaciji \url{https://localhost/aplikacija}. 
Za povezivanje nginx-a i python-flask aplikacije koristite uwsgi program. \\

Kao rješenje, bit će potrebno pokazati sve korištene konfiguracijske datoteke i eventualne dodatne skripte koje ste napisali.

\end{document}
