\documentclass[12pt,a4paper]{article}

\usepackage[croatian]{babel}
\usepackage[utf8]{inputenc}

\usepackage[margin=2cm]{geometry}
\pagenumbering{gobble}

\begin{document}
	\title{Laboratorijska vježba 1\\{\large Napredno korištenje operacijskog sustava Linux}}
	\date{\vspace{-5ex} 14.03.2015.}
	\maketitle
	John svom prezimenjaku Sherlocku svako malo treba poslati više datoteka povećeg sadržaja putem e-pošte. Oboje imaju GMail račun. Međutim, usluga ima ograničenje od 25 MB po privitku. Srećom, Holmesi su vrsni Linuxaši i dosjetili su se načina kako zaobići ograničenje.
	
	\section*{Zadatak}
	Napisati Bash skriptu koja će izračunati veličinu direktorija predanog kao argument skripti te ga kopirati na LVM logički volumen formatiran na ext4 file system.\\
	Logički volumen zauzima 100\% volumenske grupe u kojoj se nalazi više fizičkih volumena. Svaki fizički volumen je loopback uređaj napravljen od datoteke veličine 25 MiB.\\
	
	\subsection*{Primjer korištenja skripte}
	\begin{verbatim}
# ./lvm-store direktorij
Direktorij "direktorij" ima 25 MiB.
Napravljena su dva loopback uređaja s datotekama:
   disk0.vol
   disk1.vol
	\end{verbatim}	
\end{document}