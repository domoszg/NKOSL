\documentclass[12pt,a4paper]{article}

\usepackage[croatian]{babel}
\usepackage[utf8]{inputenc}

\usepackage[margin=2cm]{geometry}
\pagenumbering{gobble}

\begin{document}
	\title{Završna laboratorijska vježba\\{\large Napredno korištenje operacijskog sustava Linux}}
	\date{\vspace{-5ex} 14.03.2015.}
	\maketitle
	\section*{Zadatak}
	Na virtualni ili fizički stroj treba instalirati Arch Linux. Konfiguracija instalacije mora biti sljedeća:
	\begin{itemize}
		\item Xorg s instaliranim openbox window managerom. Grafičko sučelje se automatski pokreće pri pokretanju računala i prikazuje login prompt.
		\item Bootloader i fstab konfigurirani prema zahtjevima iz prve DZ
		\item Instalirana podrška za zvuk s GUI alatima za podešavanje glasnoće
		\item Pristup internetu preko eduroam bežične ili žičane mreže konfiguriran pomoću wpa\_supplicant
		\item[] {\small Ako tehnički ne bude moguće spajanje na mrežu (loša pokrivenost, problemi s mrežnim sučeljem na virtualnom stroju, ...) morate objasniti konfiguraciju mrežnih postavki. Lozinku uklonite iz konfiguracije.}
		\item Upravitelj paketa konfiguriran za dohvaćanje sa servera u Hrvatskoj
		\item Sat sinkroniziran s NTP serverom u Hrvatskoj\\
	\end{itemize}
	Kod instalacije na virtualnom stroju alati za virtualni stroj moraju biti instalirani i podešeni, a kod instalacije na fizičkom stroju sva periferija mora ispravno raditi.\\
	\,\\
	Rješenja trebate demonstrirati na računalu u terminu laboratorijskih vježbi. Prethodno ne trebate ništa predavati.

\end{document}