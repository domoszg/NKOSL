\documentclass[12pt,a4paper]{article}

\usepackage[croatian]{babel}
\usepackage[utf8]{inputenc}

\usepackage[margin=2cm]{geometry}
\usepackage[colorlinks=true,urlcolor=black]{hyperref}
\pagenumbering{gobble}

\renewcommand*{\familydefault}{\sfdefault}
\renewcommand*{\sfdefault}{lmss}

\begin{document}
	\title{Domaća zadaća 2}
	\date{\vspace{-5ex} 8.5.2016.}
	\maketitle

\noindent Za ovu zadaću potrebno je imati instaliranu bilo koju Linux distribuciju koja koristi systemd i netctl.

\section*{Zadatak 1}

Potrebno je napisati skriptu \texttt{ping-log.sh} koja će svakih 10 sekundi pokrenuti naredbu \\
\indent \texttt{ping -c 1 8.8.8.8}\\
Ako se na naredbu vrati ping reply potrebno je u datoteku \texttt{/var/log/nkosl/ping.log} zapisati cijelu liniju koju naredba ispisuje na stdout. Ako ping naredba iz bilo kojeg razloga ne dobije odgovor, informaciju o tomu treba zapisati u systemd journal. Na početku svakog zapisa mora postojati timestamp u formatu koji je dan primjerom: \\
\indent \texttt{[2015-05-15 13:37:00+02:00]}\\
Napišite systemd service file za skriptu i omogućite njeno pokretanje i prekidanje kroz systemd.\\

Napisati skriptu \texttt{tcpdump-log.sh} koja će na mrežnom uređaju pratiti i bilježiti pakete koje šalje ping naredba u datoteku \texttt{/var/log/nkosl/tcpdump.log}. Pratiti samo pakete poslane na odredište iz \texttt{ping-log.sh} skripte. Nije potrebno konfigurirati logrotate. Napišite systemd service file za skriptu i omogućite njeno pokretanje i prekidanje kroz systemd.\\

Kao rješenje zadatka pošaljite tar arhivu sa svim korištenim skriptama i očuvanim putanjama u odnosu na \texttt{/}.

\section*{Zadatak 2}

Pripremite \texttt{netctl} profile za pristup zadanim mrežama
\begin{enumerate}
	\item DHCP žičani pristup
	\item Bežični pristup proizvoljnoj mreži s WPA PSK autentifikacijom
	\item eduroam bežični pristup
	\item eduroam žičani pristup\footnote{Konfiguracija žičanog eduroam pristupa neće se ocjenjivati, no predlažemo da je obavite za vlastitu vježbu. Koriste se isti autentifikacijski parametri kao i kod bežične mreže, a funkcionalnost možete provjeriti u studentskim domovima ili na drugim sastavnicama Sveučilišta (npr. knjižnica FFZG) gdje postoji ovakav pristup.}
\end{enumerate}
Za eduroam pristup nije dozvoljeno ručno konfigurirati \texttt{wpa\_supplicant}. Lozinke smijete izostaviti iz konačne konfiguracije, no preporučamo korištenje hash zapisa. Potrebno je koristiti provjeru identiteta servera certifikatom koji je dostupan na \url{http://installer.eduroam.hr}.\\

Kao rješenje zadatka pošaljite tar arhivu s netctl profilima.

\end{document}