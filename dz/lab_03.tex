\documentclass[12pt,a4paper]{article}

\usepackage[croatian]{babel}
\usepackage[utf8]{inputenc}

\usepackage[margin=2cm]{geometry}
\pagenumbering{gobble}

\begin{document}
	\title{Laboratorijska vježba 2\\{\large Napredno korištenje operacijskog sustava Linux}}
	\date{\vspace{-5ex} 30.03.2015.}
	\maketitle
	
	U tvrtki Sevilla Ltd. se zaposlio Randy Spears. Ubrzo je zamijetio Bryanovo rješenje problema s mrežom. Odlučio je uz Bryanovu pomoć napraviti sustav koji ima spremnu konfiguraciju i koji kod svakog gašenja zapiše svoje postavke u log datoteku.
	
	\section*{Zadatak}
	
	Napisati Bash skriptu \texttt{generate-system.sh} koja će napraviti prazan disk image, formatirati ga na ext4 datotečni sustav i na njemu kreirati osnovni Debian testing sustav 32-bitne arhitekture. Skripta mora na novom sustavu podesiti mrežne postavke tako da odgovaraju onima iz 2. domaće zadaće.\\
	
	Napisati Bash skriptu \texttt{logger.sh} koja na kraj datoteke \texttt{/var/log/nkosl/shutdown.log} na novom sustavu dopisuje sljedeće informacije: 
	\begin{itemize}
		\item verzija kernela
  		\item popis učitanih kernel modula s njihovom putanjom
  		\item svi mrežni uređaji s pripadajućom IP adresom i maskom u CIDR notaciji
  		\item ispis sadržaja iptables-a
  	\end{itemize}
	Skripta \texttt{generate-system.sh} mora osigurati da se \texttt{logger.sh} pokreće svaki put kada gasimo računalo. Put do \texttt{logger.sh} skripte se predaje u argumentu.\\
	
	Kao rješenje šaljete tar arhivu sa skriptama \texttt{generate-system.sh} i \texttt{logger.sh}. Ako ste koristili dodatne skripte i njih uključite u tar arhivu.
	
	\subsection*{Primjer korištenja skripte}
	\begin{verbatim}
# ./generate-system.sh logger.sh
	\end{verbatim}	
\end{document}