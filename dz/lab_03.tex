\documentclass[12pt,a4paper]{article}

\usepackage[croatian]{babel}
\usepackage[utf8]{inputenc}

\usepackage[margin=2cm]{geometry}
\pagenumbering{gobble}

\begin{document}
	\title{Laboratorijska vježba 2\\{\large Napredno korištenje operacijskog sustava Linux}}
	\date{\vspace{-5ex} 15.05.2015.}
	\maketitle
	
        Novopromovirani manageri, Sunny Leone i Peter North, dobili su zadatak smanjenja troškova proizvodnje. Kao prvi korak, odlučili su virtualizirati postojeće serverse resurse.
        Vaš je zadatak pomoći im da se odluče za jednostavnije rješenje: Docker platforma ili Vagrant.
        Vagrant jest wrapper software oko VirtualBox-a.
        
	
	\section*{Zadatak}
	
        Docker platforma \\
        
        Potrebno je virtualizirati web aplikaciju iz 1. laboratorijske vježbe - python HelloWorldServer.\\
        Aplikaciju je potrebno upogoniti u vlastitom docker containeru. Početnu sliku (eng. image) za container odaberite po želji, također korištenje Dockerfile datoteke je opcionalno.
        Jedina iznimka u python skripti je da HelloWorldServer aplikacija MORA slušati za konekcije na portu 80. (Trenutno sluša na 8888)
        
        Način povezivanja containera i nginx web poslužitelja ostavljeno je vama na odluku.\\
        Opcionalni zadatak: automatizirati podizanje dodatnih HelloWorldServer containera i osvježavanje nginx konfiguracije.

        \section*{Zadatak}

        VirtualBox i Vagrant \\

        Potrebno je ostvariti pokretanje i pristupanje u virtualiziranom operacijskom sustavu kroz terminal, koristeći naredbe: vagrant up i vagrant ssh.
        Način na koji ćete to ostvariti ostavljeno je vama na odluku. \\
        \\
        Savjet: \\
        Instalirajte VirtualBox i Vagrant sa stranica proizvođača programa. \\
        OS za virtualizaciju ili gotovu vagrant sliku odaberite po želji. \\
        Vagrant up naredba pokreće operacijski sustav, a vagrant ssh pristupa virtualiziranom OS-u ssh protokolom. \\

\end{document}
