\documentclass[12pt,a4paper]{article}

\usepackage[croatian]{babel}
\usepackage[utf8]{inputenc}
\usepackage[symbol]{footmisc}

\usepackage[margin=2cm]{geometry}
\pagenumbering{gobble}

\begin{document}
	\title{Laboratorijska vježba 3\\{\large Napredno korištenje operacijskog sustava Linux}}
	\date{\vspace{-5ex} 22.05.2015.}
	\maketitle
	
        Novopromovirani manageri, Sunny Leone i Peter North, dobili su zadatak smanjenja troškova proizvodnje. Kao prvi korak, odlučili su virtualizirati postojeće serverse resurse. Vaš je zadatak pomoći im da se odluče za jednostavnije rješenje: Docker platforma ili Vagrant (wrapper oko VirtualBoxa).
        
	
	\section*{Zadatak}
	
        \textbf{Docker platforma} \\
        
        Potrebno je virtualizirati web aplikaciju iz 1. laboratorijske vježbe - python HelloWorldServer. Aplikaciju je potrebno upogoniti u vlastitom docker containeru. Po želji odaberite i preuzmite početnu sliku (engl. \textit{image}) containera. Korištenje Dockerfile datoteke je opcionalno. Izmijenite python skriptu tako da HelloWorldServer aplikacija sluša na portu 80. (Trenutno sluša na 8888.) Način povezivanja containera i nginx web poslužitelja ostavljen je vama na odluku.\\
        
        \footnote[1]{Zadatak označen zvjezdicom nije nužno rješiti, no može donijeti dodatne bodove na laboratorijskoj vježbi.} Automatizirati podizanje dodatnih HelloWorldServer containera i osvježavanje nginx konfiguracije.

    \section*{Zadatak}

        \textbf{VirtualBox i Vagrant} \\

        Instalirajte VirtualBox i Vagrant sa stranica proizvođača programa. Potrebno je ostvariti pokretanje i pristupanje virtualiziranom operacijskom sustavu kroz terminal koristeći naredbe: \texttt{vagrant up} i \texttt{vagrant ssh}. Po želji odaberite koju ćete distribuciju virtualizirati. Možete koristiti gotove vagrant slike (engl. \textit{images}).
        

\end{document}
