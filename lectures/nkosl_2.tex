% The "t" in documentclass vertically aligns the frame text to the top
% t-top c-center b-bottom 
\listfiles
\documentclass[croatian,t]{beamer} % must specify a language if using babel
\usetheme{CambridgeUS}
\usecolortheme{beaver}
\usepackage[utf8]{inputenc}
\usepackage{verbatim}
\usepackage{listings}
\usepackage{babel} % đ doesn't work without this package
\usepackage{datetime}
% Changing of bullet foreground color not possible if {itemize item}[ball]
\DefineNamedColor{named}{BrickRed}{cmyk}{0,0.89,0.94,0.28}
\setbeamertemplate{itemize item}[triangle]
\setbeamercolor{title}{fg=BrickRed}
\setbeamercolor{itemize item}{fg=BrickRed}
\setbeamercolor{section number projected}{bg=BrickRed,fg=white}
\setbeamercolor{subsection number projected}{bg=BrickRed}
% Beamer defines "\Tiny" so we have to redefine it to "\tiny"
\let\Tiny=\tiny

\renewcommand{\dateseparator}{.}
\newcommand{\todayiso}{\twodigit\day \dateseparator \twodigit\month \dateseparator \the\year}
\title[NKOSL]{Napredno korištenje operacijskog sustava Linux}
\subtitle{Particije i datotečni sustavi}
\author{Goran Cetušić}
\author[Goran Cetušić]{Goran Cetušić\\{\small Nositelj: dr. sc. Stjepan Groš}}
\institute[FER]{Sveučilište u Zagrebu \\
				Fakultet elektrotehnike i računarstva}
\date{\todayiso}

\begin{document}
    %\beamerdefaultoverlayspecification{<+->}
    {
    \setbeamertemplate{headline}[] % still there but empty
    \setbeamertemplate{footline}{}
    \begin{frame}
        \maketitle
    \end{frame}
    }
    
    \begin{frame}
        \tableofcontents
    \end{frame}
    
    \section{Particioniranje}
    \begin{frame}{Particije}
    	\begin{itemize}
    		\item Kako Linux prikazuje diskove
    		\begin{itemize}
    			\item diskovi idu redom po abecedi – a, b, c, d\ldots
    			\item sda, sdb\ldots SATA disk a, SATA disk b
    		\end{itemize}
    		\item Kako Linux prikazuje particije
    		\begin{itemize}
    			\item Brojanje particija kreće od 1
    			\item Primarne particije imaju brojeve od 1 do 4
    			\item Logičke particije imaju brojeve od 5 na dalje
    			\item sda1, sda2, sda5\ldots
    		\end{itemize}
    		\item Disk kontroler pristupa sektoru na disku
    	 	\item Sektor je najmanji mogući zapis podataka
    	 	\begin{itemize}
    	 		\item 512 bajta na današnjim diskovima
    	 	\end{itemize} 
    	 	\item 16 GB datoteka ne stane na jedan sektor
    	 	\begin{itemize}
    	 		\item Potrebno je nekoliko stotina sektora diska za veće datoteke
    	 		\item Često nije moguće zapisati datoteku na uzastopne sektore već su podaci zapisani na nekoliko mjesta
    	 	\end{itemize}
    	\end{itemize}
	\end{frame}
	
	\begin{frame}{Blokovi}
		\begin{itemize}
			\item Svaki datotečni sustav vodi bilješke o datotekama i posjeduje superblock – opisnik datotečnog sustava
			\item superblock sadrži sljedeće podatke o datotečnom sustavu
			\begin{itemize}
				\item tip sustava
				\item veličina sustava
				\item status
				\item informacije o metapodacima
			\end{itemize}
			\item Linux podržava veliki broj datotečnih sustava - ext4, xfs, reiserfs\ldots
			\begin{itemize}
				\item Svaki sustav na svoj način vodi evidenciju o datotekama
				\item Moguće je dodavanje podrške za dodatne sustave
				\item Lista podržanih sustava nalazi se u datoteci /proc/filesystems
			\end{itemize}
			\item Za smanjenje broja zapisa datotečni sustav koristi blokove kao mjerne jedinice
			\begin{itemize}
				\item 4KB – zahtjev kontroleru za 8 uzastopnih sektora za zapis podataka
			\end{itemize}
		\end{itemize}
	\end{frame}	
	
	\begin{frame}[fragile]
	\frametitle{Raspored particija}
		\begin{itemize}
			\item Standardni sustav ima particije podijeljene na
			\begin{itemize}
				\item / - vršni direktorij
				\item /home - matični direktoriji svih korisnika
				\item swap - particija korištena za straničenje (paging)
			\end{itemize}
			\item Ovakav raspored omogućuje očuvanje osobnih podataka kod reinstalacije sustava
			\item Za particioniranje koristimo naredbe \textit{fdisk} i \textit{parted}
			\begin{itemize}
				\item Manipuliraju informacijama o particijama u MBR-u
				\item Primjer: ispis particija na disku
				\begin{lstlisting}
				fdisk -l
				\end{lstlisting}
			\end{itemize}
			\item \textit{fdisk} ispisuje i opće podatke od disku
			\begin{itemize}
				\item Uređaj, kapacitet, emulirana CHS vrijednost
				\item tip i veličina particija\ldots
			\end{itemize}
		\end{itemize}
	\end{frame}
	
	\begin{frame}[fragile]
	\frametitle{fdisk}
		\begin{itemize}
		\item \textit{fdisk} posjeduje interaktivni mod
		\begin{lstlisting}[basicstyle={\footnotesize\ttfamily},language=bash]
     p Ispisuje particijsku tablicu
     n Dodaje novu particiju
     t Mijenja tip particije
     d Brise particiju
     w Zapisuje novu tablicu particija i napusta fdisk
     q Izlazi bez zapisivanja nove tablice particija
		\end{lstlisting}
		\item Stvaranje particije sastoji se od sljedećih koraka
		\begin{itemize}
			\item pokretanje \textit{fdisk} naredbe
			\item stvaranje nove particije (primarne ili proširene)
			\item upisivanje početnog sektora
			\item upisivanje završnog sektora
			\item postavljanje tipa particije (Linux, swap, RAID\ldots)
			\item zapisivanje postavki
		\end{itemize}
		\end{itemize}
	\end{frame}
	
	\begin{frame}[fragile]
	\frametitle{parted}
		\begin{itemize}
			\item Po upotrebi je naredba \textit{parted} identična naredbi \textit{fdisk}
			\item \textit{parted} podržava GPT  
		\begin{lstlisting}[basicstyle={\tiny\ttfamily},language=bash]
# parted -l
Model: ATA HITACHI HTS72323 (scsi)
Disk /dev/sda: 320GB
Sector size (logical/physical): 512B/512B
Partition Table: msdos

Number  Start   End     Size    Type     File system     Flags
 1      1049kB  18.1GB  18.1GB  primary  ext4            boot
 3      18.1GB  20.1GB  1994MB  primary  linux-swap(v1)
 2      20.1GB  311GB   291GB   primary  ext4
 4      311GB   320GB   8984MB  primary  ext4
		\end{lstlisting}
			\item Naredba \textit{parted} ne nudi mogućnost mijenjanja tipa particije (koda)
			\begin{itemize}
				\item mijenja se tijekom stvaranje particija ili instalacijom datotečnog sustava
				\item u ovom slučaju je naredba \textit{fdisk} fleksibilnija
			\end{itemize}
		\begin{lstlisting}[basicstyle={\tiny\ttfamily},language=bash]
# parted /dev/sda
GNU Parted 2.3
Using /dev/sda
Welcome to GNU Parted! Type 'help' to view a list of commands.
(parted) help
    .
    .
    .
		\end{lstlisting}
		\end{itemize}
	\end{frame}	
	
	\section{Stvaranje datotečnog sustava}
	\begin{frame}[fragile]
	\frametitle{dd, mkfs}
		\begin{itemize}
			\item Linux sustavi najčešće koriste ext derivate 
			\begin{itemize}
				\item ext2, ext3, ext4
				\item Veliki dodatak je \textit{journaling} promjena uz zapisivanje podataka
			\end{itemize}
			\item Za primjere će umjesto diska biti korištena datoteka
			\item Prvo je potrebno stvoriti praznu datoteku
			\begin{itemize}
				\item Koristi se naredba \textit{dd} za stvaranje datoteke
				\begin{lstlisting}[basicstyle={\footnotesize\ttfamily},language=bash]
				dd if=/dev/zero bs=4k count=8000 of=filesystem
				\end{lstlisting}
				\item Ovime je stvorena datoteka popunjena nulama veličine 32MB
			\end{itemize}
			\item Sljedeći korak je stvaranje datotečnog sustava
			\item Naredba \textit{mkfs} je frontend za \textit{mkfs.ext4, mkfs.xfs} itd.
			\begin{lstlisting}[basicstyle={\footnotesize\ttfamily},language=bash]
			mkfs -t ext4 filesystem
			\end{lstlisting}
			\item Datoteka sada posjeduje superblock
		\end{itemize}
	\end{frame}
	
	\begin{frame}[fragile]
	\frametitle{Swap}
		\begin{itemize}
			\item Linux koristi swap particiju za paging – spremanje procesa na disk
			\item swap particija/datoteka je obavezna pri instalaciji sustava
			\item Swap particija se jednako montira kao ostale particije ali se posebno stvara i uključuje
			\begin{itemize}
				\item stvaranje swap particije
				\begin{lstlisting}[basicstyle={\footnotesize\ttfamily},language=bash]
				mkswap <device>
				\end{lstlisting}
				\item isključivanje i uključivanje particije
				\begin{lstlisting}[basicstyle={\footnotesize\ttfamily},language=bash]
				swapon/swapoff
				\end{lstlisting}
			\end{itemize}
			\item U verzijama 2.6 i 3.x kernela, moguće je umjesto swap particije koristiti swap datoteku
			\begin{itemize}
				\item Kernel izravno a ne preko datotečnog sustava vodi evidenciju o položaju datoteke, time se smanjuje brzina čitanja
				\item Moguć je neograničen broj swap datoteka
			\end{itemize}
		\end{itemize}
	\end{frame}
	
	\begin{frame}[fragile]
	\frametitle{dumpe2fs, tune2fs, fsck}
		\begin{itemize}
			\item Na sustavu se rade njegove kopije superblocka jer je superblock zapis koji posjeduje glavne informacije o datotekama
			\item \textit{dumpe2fs} Ispisuje postavke \textit{ext} datotečnih sustava
			\begin{lstlisting}[basicstyle={\footnotesize\ttfamily},language=bash]
			dumpe2fs -h filesystem
			\end{lstlisting}
			\item Postavke se mogu promijeniti
			\begin{itemize}
				\item Primjer: postavljanje labele
				\begin{lstlisting}[basicstyle={\footnotesize\ttfamily},language=bash]
				tune2fs -L nkosl_image filesystem
				\end{lstlisting}
			\end{itemize}
			\item Konzistentnost sustava je moguće narušiti tijekom ispada poput nestanka struje a naredba \textit{fsck} može popraviti sustav
			\begin{lstlisting}[basicstyle={\footnotesize\ttfamily},language=bash]
			fsck -y filesystem
			\end{lstlisting}
			\item Distribucije periodički rade provjeru sustava kod pokretanja
		\end{itemize}
	\end{frame}
	
	\begin{frame}{Inode}
		\begin{itemize}
			\item Inode je opisnik datoteka na Unix sustavima
			\begin{itemize}
			\item Tip datoteke
			\item Dopuštenja
			\item Vlasnik datoteke
			\item Grupa datoteke
			\item Veličina podataka
			\item Vrijeme zadnje modifikacije, pristupa i mijenjanja (engl. MACtimes)
			\item Broj tvrdih poveznica (engl. hard links)
			\item Atributi
			\item Kontrolne liste - ACL (engl. Access Control Lists)
			\item Struktura sa lokacijom blokova podataka
			\end{itemize}
			\item Podaci se gotovo nikada ne zapisuju na uzastopne sektore već se nalaze na različitim dijelovima diska
			\item Datotečni sustav pomoću pokazivača na blokove može dohvatiti sve dijelove datoteke
		\end{itemize}
	\end{frame}
	
	\section{Montiranje}
	\begin{frame}[fragile]
	\frametitle{mount}
		\begin{itemize}
			\item Datotečnom sustavu nije moguće pristupiti prije montiranja tj. dodavanja u hijerarhiju datotečnog sustava
			\begin{itemize}
			\begin{lstlisting}[basicstyle={\footnotesize\ttfamily},language=bash]
			mount -t <fstype> <device> <mountpoint>
			\end{lstlisting}
				\item ovime je na sustav montiran blok uređaj
				\item ponekad naredba \textit{mount} sama prepoznaje vrstu datotečnog sustava pa opcija \textit{-t} nije potrebna
			\end{itemize}
			\item Montirati se mogu samo blok uređaji
			\item Poseban modul omogućuje montiranje datoteka kao blok uređaja
			\begin{lstlisting}[basicstyle={\footnotesize\ttfamily},language=bash]
			modprobe loop
			\end{lstlisting}
			\item naredba \textit{mount} koristi \textit{loop} modul
			\begin{lstlisting}[basicstyle={\footnotesize\ttfamily},language=bash]
			mount -o loop <fstype> <device> <mountpoint>
			\end{lstlisting}
			\item Montirani pseudouređaji imaju naziv /dev/loopX, gdje je X broj 0-7
		\end{itemize}
	\end{frame}	
	
	\begin{frame}[fragile]
		\begin{itemize}
			\item Neki parametri naredbe \textit{mount} (nastavak)
			\begin{itemize}
				\item Sprečavanje pisanja na particiju
				\begin{lstlisting}[basicstyle={\footnotesize\ttfamily},language=bash]
					mount -o ro <particija> <direktorij>
				\end{lstlisting}
				\item Zabrana izvršavanja programa
				\begin{lstlisting}[basicstyle={\footnotesize\ttfamily},language=bash]
					mount -o noexec <particija> <direktorij>
				\end{lstlisting}
				\item Promjena parametara ako je particija već montirana
				\begin{lstlisting}[basicstyle={\footnotesize\ttfamily},language=bash]
					mount -o remount,noexec <particija>
				\end{lstlisting}
			\end{itemize}
			\item Što ako montiramo istu particiju više puta i/ili na istu lokaciju?
			\item Postoji nekoliko kombinacija
			\begin{itemize}
				\item Ista particija na različite lokacije
				\item Ista particija na iste lokacije
				\item Različite particije na iste lokacije
				\item Različite particije na različite lokacije
			\end{itemize}
			\item Zadatak: Isprobajte sve kombinacije
		\end{itemize}
	\end{frame}
	
	\begin{frame}[fragile]
	\frametitle{fstab}
		\begin{itemize}
			\item Montiranje je privremeno, vrijedi do sljedećeg gašenja računala ali je moguća automatizacija
			\item Datoteka \textit{/etc/fstab} sadrži zapise particija sa datotečnim sustavima koje je potrebno montirati
			\begin{itemize}
				\item Također sadrži točke montiranja, tip datotečnog sustava i parametre
			\end{itemize}
			\item Svaka linija predstavlja jedan datotečni sustav
			\begin{lstlisting}[basicstyle={\scriptsize\ttfamily},language=bash]
# <file system> <mount point>   <type>  <options>  <dump>  <pass>
proc            /proc           proc    defaults   0       0
			\end{lstlisting}
			\item Naredba \textit{mount} bez argumenata ispisuje trenutno montirane uređaje
			\item Napomena: nije moguće "montirati uređaje", samo datotečne sustave sa superblock zapisom
			\begin{itemize}
				\item Što ima smisla jer je superblock obilježje datotečnog sustava
				\item Primjer: Nije moguće montirati disk /dev/sda ali je moguće jednu od njegovih particija /dev/sda1 sa ext4 sustavom
			\end{itemize}
		\end{itemize}
	\end{frame}
	
	\begin{frame}{UUID}
		\begin{itemize}
			\item UUID se u datoteci /etc/fstab koristi kao jedinstven identifikator datotečnog sustava
			\item Ako je navedeno ime datoteke uređaja, njeno ime nije jedinstveno na svakom sustavu
			\item Primjer: Četiri identična SATA diska sa particijama i datotečnim sustavima su priključeni na matičnu ploču. U /etc/fstab datoteci su navedeni kao sdaX, sdbX, sdcX, sdeX gdje su X brojevi koji određuju particije. Ako zamijenimo matičnu ploču i ponovno priključimo diskove, postoji mogućnost da ih operacijski sustav prepozna drugim redoslijedom: sda je sada sdc, a sdb je sde.
			\begin{itemize}
				\item Sada je potrebno mijenjati /etc/fstab datoteku
				\item Pogoditi ispravan raspored diskova može biti problem
			\end{itemize} 
			\item Problem se mogao izbjeći korištenjem UUID identifikatora umjesto imena datoteka
		\end{itemize}
	\end{frame}

	\begin{frame}[fragile]
	\frametitle{umount}
		\begin{itemize}
			\item \textit{umount} odmontira datotečni sustav iz hijerarhije
			\item Moguće je odmontirati po uređaju/particiji ili lokaciji
			\begin{lstlisting}
				umount /mnt
				umount /dev/sda1
			\end{lstlisting}
			\item Kada direktorij/sustav nije moguće odmontirati?
			\begin{itemize}
				\item Direktorij nije moguće odmontirati ako mu pristupa proces
				\item Direktorij nije moguće odmontirati ni ako je tekući direktorij onaj koji pokušavamo odmontirati
			\end{itemize}
			\item U drugom slučaju je potrebno izaći iz direktorija
			\item U prvom je moguće pogledati koji proces pristupa direktoriju i terminirati ga ili pričekati da završi pomoću lazy opcije
			\begin{lstlisting}
				umount -l <mountpoint>
			\end{lstlisting}
		\end{itemize}
	\end{frame}	
	
	\section{Bootloader}
	\begin{frame}{GRUB/LILO}
		\begin{itemize}
			\item Bootloader je zadnja faza prije učitavanja operacijskog sustava u memoriju; on je zadužen za učitavanja OS-a
			\begin{itemize}
				\item CPU se pokrene i učita BIOS. BIOS učita MBR u kojem se nalazi \textit{bootloader}. Bootloader pročita zapis o lokaciji kernela i učita ga u memoriju. Operacijski sustav nakon toga ima potpuni nadzor.
			\end{itemize}
			\item Najčešće korišteni \emph{bootloaderi} su 
			\begin{itemize}
				\item GRUB - GRand Unified Bootloader
				\item LILO - LInux LOader
			\end{itemize}
			\item Mi ćemo se baviti GRUB-om, standardnim bootloaderom na Debian, Ubuntu, Arch, Fedora i ostalim distribucijama 
		\end{itemize}
	\end{frame}	
	
	\begin{frame}{GRUB}
		\begin{itemize}
			\item Dvije verzije GRUB bootloadera su
			\begin{itemize}
				\item GRUB 2
				\item GRUB - izlaskom GRUB 2 preimenovan u GRUB Legacy
			\end{itemize}
			\item GRUB 2 potpuno mijenja strukturu vlastitih konfiguracijskih datoteka
			\item Najvažnije datoteke (lokacije mogu ovisiti o distribuciji):
			\begin{itemize}
				\item /boot/grub/grub.cfg
				\item /etc/default/grub
				\item /etc/grub.d/ (direktorij)
			\end{itemize}
			\item Naredba update-grub pročita /etc/default/grub i datoteke unutar grub.d direktorija te izgenerira datoteku grub.cfg
			\begin{itemize}
				\item bootloader čita grub.cfg tijekom učitavanja sustava
				\item grub.cfg nije namijenjena da bude ručno modificirana
			\end{itemize}
		\end{itemize}
	\end{frame}
	
	\begin{frame}[fragile]
	\frametitle{hdX}
		\begin{itemize}
			\item GRUB na različit način od operacijskog sustava označava uređaje
			\item Svi diskovi su oblika hdX,Y
			\begin{itemize}
				\item X je oznaka diska, počinje od 0
				\item Y je oznaka particije, počinje od 1 (Legacy broji od 0)
			\end{itemize}
			\item GRUB ponudi popis dostupnih "aplikacija" (ne moraju svi biti operacijski sustavi)
			\item Primjer jedne stavke u konfiguracijskoj datoteci
			\begin{lstlisting}[basicstyle={\scriptsize\ttfamily},language=bash]
menuentry "Ubuntu_10.04" {
set root=(hd0,1)
search --no-floppy --fs-uuid --set cb201140-52f8-4449-9a95-749b27b58ce8
linux /boot/vmlinuz-2.6.31-11-generic root=UUID=cb201140-52f8-4449-9a95-749b27b58ce8 ro quiet splash
initrd /boot/initrd.img-2.6.31-11-generic
}
			\end{lstlisting}		
		\end{itemize}
	\end{frame}
	
	\begin{frame}[fragile]
	\frametitle{root, linux, initrd}
		\begin{itemize}
			\item GRUB zahtijeva nekoliko informacija kako bi uspješno podigao sustav
			\begin{itemize}
				\item root - particija na kojoj se nalazi operacijski sustav
				\item linux - lokacija datoteke kernela sa parametrima
				\item initrd - lokacija modula za prepoznavanje potrebnih uređaja
			\end{itemize}
			\item GRUB podržava nekoliko operacijskih sustava ali među njima nisu Microsoft Windows
			\item Moguće je odgovornost prebaciti na drugi bootloader - \emph{chainloading}
			\begin{itemize}
				\item Umjesto potpunih informacija navede se lokacija drugog bootloadera
			\end{itemize}
			\item Ova metoda se koristi za podizanje Windows sustava
			\begin{lstlisting}[basicstyle={\scriptsize\ttfamily},language=bash]
menuentry "Windows_7" {
set root=(hd0,8)
chainloader +1
}
			\end{lstlisting}
		\item Naredba \emph{chainloader} prebaci odgovornost na bootloader koji se nalazi na prvom sektoru particije (+1)
		\end{itemize}
	\end{frame}

	\begin{frame}[fragile]
	\frametitle{GRUB 2 vs GRUB Legacy}
		\begin{itemize}
			\item Naredbe (GRUB 2)
			\begin{itemize}
				\item grub-mkconfig
				\item grub-update
				\item grub-install
				\item grub-mkdevicemap
				\item ...
			\end{itemize}
			\item Podrška za LVM i RAID
			\item Funkcionalnost je podijeljena po modulima
			\begin{itemize}
				\item moduli se nalaze u direktoriju /boot/grub
			\end{itemize}
			\item Datoteke potrebne za podizanje sustava su reorganizirane
			\begin{itemize}
				\item boot.img, diskboot.img core.img... (GRUB 2)
				\item stage1, stage1.5, stage2 (GRUB Legacy)
			\end{itemize}
		\end{itemize}
	\end{frame}
	
	\section{}
	\begin{frame}{Literatura}
		\begin{tiny}
			\url{http://www.cyberciti.biz/tips/understanding-unixlinux-file-system-part-i.html} \\
			\url{http://www.angelfire.com/myband/binusoman/Unix.html} \\
			\url{http://sakafi.wordpress.com/2008/08/23/how-to-use-parted-for-creating-patition-larger-that-2-tb/} \\
			\url{http://liquidat.wordpress.com/2007/10/15/short-tip-get-uuid-of-hard-disks/} \\
			\url{https://help.ubuntu.com/community/Grub2} \\
			\url{http://www.gnu.org/software/grub/manual/html_node/chainloader.html} \\
			\url{http://www.gnu.org/software/grub/manual/grub.html} \\
		\end{tiny}
	\end{frame}
\end{document}
